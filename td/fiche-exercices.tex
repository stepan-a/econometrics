\documentclass[10pt,a4paper,notitlepage]{article}
\usepackage{amsmath}
\usepackage{amssymb}
\usepackage{amsbsy}
\usepackage{bbm}
\usepackage{bm}
\usepackage{float}
\usepackage[french]{babel}
\usepackage{graphicx}
\usepackage{enumerate}

%\usepackage{palatino}

 \usepackage[active]{srcltx}
\usepackage{scrtime}

\newcounter{xnumber}
\setcounter{xnumber}{0}

\newcommand{\exercice}{\textsc{\textbf{Exercice}} \textbf{\addtocounter{xnumber}{1}\thexnumber}\,\,}
\newcommand{\question}[1]{\textbf{(#1)}}
\setlength{\parindent}{0cm}

\begin{document}

\title{\textsc{Économétrie approfondie\\ \small{(Exercices)}}}
\author{Stéphane Adjemian\thanks{Université du Mans. \texttt{stephane DOT adjemian AT univ DASH lemans DOT fr}}}
\date{Le \today\ à \thistime}

\maketitle

\exercice La nature génère des données avec le modèle suivant~:
\[
y_t = \mu_1 + \mu_2 \mathbbm{1}_{\{t>\kappa\}}(t) + \varepsilon_t
\]
avec $\mu_1$ et $\mu_2$ des paramètres réels,
$\varepsilon_t\sim\mathrm{iid}(0,\sigma_{\varepsilon}^2)$, $\kappa$ un entier donné entre  1 and $T$. Le modèle empirique est~:
\[
  y_t = c + u_t
\]
On note $\mathcal Y_T=\{y_1,\dots,y_T\}$ l'échantillon produit par la nature et
à disposition de l'économètre. \question{1} Donner l'expression de l'estimateur
des MCO de $c$, noté $\hat c_T$. \question{2} Calculer l'espérance de
$\hat c_T$, en explicitant le rapport avec les paramètres du modèle de la
nature. \question{3} Calculer la variance de $\hat c_T$. (4) Quel est la limite
en probabilité de $\hat c_T$~?\newline

\bigskip

\exercice La nature génère des données avec le modèle suivant~:
\[
y_t = \mu_1 + \mu_2 \mathbbm{1}_{\{t>[\kappa T]\}}(t) + \varepsilon_t
\]
avec $\mu_1$ et $\mu_2$ des paramètres réels,
$\varepsilon_t\sim\mathrm{iid}(0,\sigma_{\varepsilon}^2)$, $\kappa\in]0,1[$, et $[x]$ représente la partie entière de $x$. Le modèle empirique est~:
\[
  y_t = c + u_t
\]
On note $\mathcal Y_T=\{y_1,\dots,y_T\}$ l'échantillon produit par la nature et
à disposition de l'économètre. Reprendre les questions de l'exercice 1 avec ce
nouveau DGP.\newline

\bigskip

\exercice Soit le modèle empirique:
\[
y_i = \beta_0 + \beta_1x_{1,i} + \beta_2 x_{2,i} + \dots + \beta_K x_{K,i} + u_i
\]
L'échantillon est $\mathcal Y_T = \{y_1, \dots, y_N, x_1, \dots, x_N\}$.
\question{1} Écrire le modèle matriciellement sous la forme $Y = X\beta+U$. \question{2} Montrer que l'estimateur des MCO est $\hat\beta = (X'X)^{-1}X'Y$. \question{3} Expliciter le contenu de $X'X$ et $X'Y$ en faisant apparaître les observations individuelles. \question{4} Dans le cas $K=1$, sur la droite de l'équation ne reste que la constante et une variable explicative $x_{1,i}$, écrire explicitement à partir des réponses aux questions (2) et (3) l'estimateur des MCO (pour la constante et la pente). En passant, montrer que le déterminant de $X'X$ est strictement positif dès lors que $N>1$ et $x_1\neq 0$. \question{5} Montrer que la somme des résidus est nulle par construction.\newline

\bigskip

\exercice La nature génère des données avec le modèle suivant~:
\[
y_t = \mu_0 + \mu_1 t  + \varepsilon_t
\]
avec $\mu_0$ et $\mu_1$ des paramètres réels, et
 $\varepsilon_t\sim\mathrm{iid}(0,\sigma_{\varepsilon}^2)$. Le modèle empirique est~:
\[
  y_t = c + u_t
\]
On note $\mathcal Y_T=\{y_1,\dots,y_T\}$ l'échantillon produit par la nature et
à disposition de l'économètre. \question{1} Donner l'expression de l'estimateur
des MCO de $c$, noté $\hat c_T$. \question{2} Calculer l'espérance de
$\hat c_T$. S'agit-il d'un estimateur non biaisé de $\mu_0$~?\newline

\bigskip

\exercice Soit la matrice carrée~:
\[
  \mathcal A =
  \begin{pmatrix}
    A_{11} & A_{12}\\
    A_{21} & A_{22}
  \end{pmatrix}
\]
où $A_{11}$ et $A_{22}$ sont des matrices carrées, $A_{11}$ et $A_{22}-A_{21}A_{11}^{-1}A_{12}$ sont des matrices de plein rang. \question{1} Montrer que~:
\[
  \mathcal A^{-1} =
  \begin{pmatrix}
    A_{11}^{-1}\left( I + A_{12}BA_{21}A_{11}^{-1}\right) & -A_{11}^{-1}A_{12}B\\
    -BA_{21}A_{11}^{-1} & B
  \end{pmatrix}
\]
avec $B=\left(A_{22}-A_{21}A_{11}^{-1}A_{12}\right)^{-1}$. \question{2} Appliquer la formule dans le cas où $A_{12}$ et $A_{21}$ sont des matrices nulles.\newline

\bigskip

\exercice Soit le modèle, sous forme matricielle, suivant~:
\[
  Y = X \beta + U
\]
où $Y$ et $U$ sont des vecteurs $N\times 1$, $X$ une matrice $N\times K$ et $\beta$ un vecteur $K\times 1$. On partitionne la matrice des variables explicatives de la façon suivante~:
\[
X = \left(X_1 \vdots X_2\right)
\]
où $X_1$ et $X_2$ sont des matrices $N \times K_1$ et $N\times K_2$ avec $K_1\geq 1$, $K_2\geq 1$ et $K_1+K_2=K$. On partitionne le vecteur des paramètres de la même façon $\beta_1$ contient les paramètres associés aux variables dans $X_1$, $\beta_2$ les paramètres associés aux variables dans $X_2$. \question{1} Donner les estimateurs des MCO de $\beta_1$ et $\beta_2$ dans le cas où les variables dans $X_1$ sont orthogonales à celles dans $X_2$ (c'est-à-dire dans le cas où $X_1'X_2 = 0$ et $X_2'X_1=0$). Interpréter. \question{2} Si les matrices $X_1$ et $X_2$ ne sont pas des matrices orthogonbales, donner une expression de $\hat\beta_1$ en la comparant à l'expression obtenue dans la question précédente. \question{3} Montrer que le vecteur des résidus estimés peut s'écrire sous la forme $\hat U = M Y$ (en donnant une expression pour la matrice $M$). \question{4} Montrer que la matrice $M$ est idempotente.\question{5} Montrer que les matrices $M$ et $X$ sont orthogonales. \question{6} Montrer que la prédiction $\hat Y = X\hat\beta$ peut s'écrire sous la forme $\hat Y = H Y$. La matrice $H$ est-elle idempotente~? \question{7} Éliminer $\hat\beta_1$ dans l'expression de $\hat\beta_2$, obtenu en répondant à la question (2), en faisant apparaître le projecteur orthogonal $M_1 = I-X_1(X_1'X_1)^{-1}X_1$. Commenter.\newline

\bigskip

\exercice On cherche à illustrer à l'aide de simulations la convergence de
l'estimateur des MCO. On considère le DGP suivant~:
\[
y_i = a + b x_i + \varepsilon_i
\]
$i=1,\dots, N$, où $\varepsilon_i$ est un bruit blanc gaussien centré réduit,
$a=1$ et $b=\frac{1}{2}$ sont les vraies valeurs des paramètres. On commencera
par supposer que la variable explicative est déterministe, c'est-à-dire que la
réalisation des $x_i$ est constante d'un échantillon à l'autre. \question{1}
Pour $N=10$ (taille de l'échantillon), simuler les $\{x_i\}_{i=1}^N$ avec une
loi normale centrée sur 1 et de variance 2. puis dans une boucle 50000
échantillons $\{y_i\}_{i=1}^N$ en estimant pour chaque échantillon les
paramètres $\alpha$ et $\beta$ dans le modèle empirique~:
\[
y_i = \alpha + \beta x_i + u_i
\]
Les 50000 valeurs obtenues pour $\widehat\beta$ devront être stockées dans un
vecteur. Calculer la moyenne et la variance des $\widehat\beta$ obtenus. Que
représentent ces quantités. \question{2} Reprendre la question (1) avec $N=100$,
$N=1000$ et $N=10000$. Comparer les moyennes et variances puis commenter.
\question{3} Reprendre les questions (1) et (2) en considérant que la variable
explicative n'est pas déterministe (pour former chaque échantillon il faut tirer
une réalisation différente des $\{x_i\}_{i=1}^N$ dans une loi normale centrée
sur 1 et de variance 2). \question{4} Reprendre la question (3) en considérant
que la variance de la variable explicative est 0.2 puis 20 (plutôt que 2).
Comment ce changement affecte les propriétés de l'estimateur de $\beta$~?\newline

\bigskip

\exercice On cherche à illustrer à l'aide de simulations la distribution de
l'estimateur des MCO. On reprend le même cadre que dans l'exercice précédent,
mais au lieu de calculer la moyenne et la variance on représente la distribution
de l'estimateur de $\beta$ en déviation à la vraie valeur du paramètre $b$,
$\widehat\beta-\frac{1}{2}$. On peut utiliser des histogrammes pour représenter
les distributions. Reprendre les mêmes scenarii que dans l'exercice précédent et
représenter les distributions de $\widehat\beta-\frac{1}{2}$ puis
$\sqrt{N}\left(\widehat\beta-\frac{1}{2}\right)$. Commenter.\newline

\bigskip

\exercice Soit un couple de variables aléatoires réelles $(X_1, X_2)$ distribué selon une loi normale bivariée~:
\[
  \begin{pmatrix}
    X_1\\
    X_2
  \end{pmatrix}
  \sim
  \mathcal N
  \left(
    \begin{pmatrix}
      \mu_1\\
      \mu_2
    \end{pmatrix},
    \begin{pmatrix}
      \sigma_1^2 & \rho\sigma_1\sigma_2\\
      \rho\sigma_1\sigma_2 & \sigma_2^2
    \end{pmatrix}
  \right)
\]
avec $(\mu_1, \mu_2)\in\mathbb R^2$, $(\sigma_1, \sigma_2)\in\mathbb R_+^2$ et
$0<\rho<1$. \question{0} Comment s'interprète le paramètre $\rho$~? \question{1}
Écrire la densité jointe du couple $(X_1, X_2)$. \question{2} Montrer qu'il est
possible de réécrire cette densité jointe comme le produit de la densité
conditionnelle de $X_1|X_2$ et de la densité marginale de $X_2$. Montrer que ces
densités sont gaussiennes. \question{3} Donner $\mathbb E[X_1|X_2]$ et
$\mathbb V[X_1|X_2]$. \question{4} Soit $X_2$ une variable aléatoire normale
$\mathcal N(\mu_2, \sigma_2^2)$. Quelle est la loi de
$X_1 = a + bX_2 + \varepsilon$ si $X_1\perp\varepsilon$ et
$\varepsilon\sim\mathcal N(0,\sigma_{\varepsilon}^2)$~? \question{5} Écrire la
densité de $X_1|X_2$. \question{6} L'utilisation du mot "régression" a été
introduite par le statisticien Sir Francis Galton dans une étude de 1885
intitulée "Regression Toward Mediocrity in Hereditary Stature" (Régression vers
la médiocrité dans la stature héréditaire). Il a montré que la taille des
enfants issus de parents très petits ou très grands se rapprochait de la
moyenne. En moyenne, l'enfant de parents de grandes tailles est plus petit.
Discuter ce résultat à l'aide de vos réponses aux questions (2) et (3).\newline

\bigskip

\exercice On considère un modèle générateur des données de la forme~:
\[
y_t = \alpha + \beta x_t + \varepsilon_t
\]
avec une variable explicative ($x$) aléatoire (à chaque fois que
l'économètre s'adresse à la nature pour obtenir un nouvel échantillon,
celle-ci retourne des $y$ \textbf{et} $x$ différents). On suppose
que $\varepsilon$ est i.i.d. et que~:
\[
  \mathbb E \left[\varepsilon_t| x_s \right] = 0
\]
\[
  \mathbb V \left[\varepsilon_t| x_s \right] = \sigma_{\varepsilon}^2
\]
pour tout $t,s$. Le modèle empirique est~:
\[
y_t = a + b x_t + \epsilon_t
\]
\question{1} Écrire l'estimateur des MCO de $b$,
noté $\hat{b}$. \question{2} Exprimer $\hat{b}$ en fonction
de $\beta$. \question{3} Soit un couple de variables aléatoires $U$
et $V$. Écrire la définition de l'espérance conditionnelle de $V|U$,
que l'on notera $\Psi(U)$. \question{4} Montrer que l'espérance
marginale de $V$ est égale à l'espérance de $\Psi(U)$, en intégrant
sur $U$. \question{5} Montrer
que $\mathbb E\left[ \varepsilon_t \right]=0$. \question{6} Montrer
que $\mathbb E\left[ \varepsilon_t x_s\right] = 0$ pour
tout $t, s$. \question{7} Montrer que $\varepsilon_t$ est
homoscédastique. \question{8} L'estimateur des MCO $\hat b$ est-il sans
biais?\newline

\bigskip

\exercice On considère un modèle générateur des données de la forme~:
\[
y_t = \alpha + \beta x_t + \varepsilon_t, \quad t=1,\dots,T
\]
avec une variable explicative ($x$) aléatoire. On suppose que la
covariance de $x$ et $\varepsilon$ est non nulle. Le modèle empirique est~:
\[
y_t = a + b x_t + \epsilon_t
\]
\question{1} L'estimateur des MCO de $b$, noté $\hat{b}$, est-il sans
biais. \question{2} Calculer la limite en probabilité de l'estimateur
des MCO. L'estimateur des MCO est-il convergent~? \question{3} On
suppose que l'on dispose des réalisations d'une autre variable
aléatoire $\{z_t, t=1,\dots,T\}$. La variable aléatoire $z$ est
corrélée avec $x$ mais non corrélée avec $\varepsilon$. On définit les
estimateurs de $a$ et $b$ par le système suivant~:
\[
  \begin{cases}
    \frac{1}{N}&\sum_{t=1}^T \left( y_t - \hat{a}_{\text{IV}} - \hat{b}_{\text{IV}} x_t \right) = 0\\
    \frac{1}{N}&\sum_{t=1}^T z_t\left( y_t - \hat{a}_{\text{IV}} - \hat{b}_{\text{IV}} x_t \right) = 0\\
  \end{cases}
\]
Justifier la définition des estimateurs. \question{4} Résoudre de
système d'équations en donnant les expressions
de $\hat{a}_{\text{IV}}$ et $\hat{b}_{\text{IV}}$. \question{5}
Montrer que $\hat{b}_{\text{IV}}$ converge en probabilité
vers $\beta$. $\hat{b}_{\text{IV}}$ est l'estimateur à variable
instrumentale de $b$. \question{6} Calculer la variance asymptotique
de $\sqrt{T}\left( \hat{b}_{\text{IV}} - \beta \right)$. Comparer avec
l'estimateur des MCO.\newline

\bigskip

\exercice On généralise l'esimateur à variable instrumentale de l'exercice précédent. Le modèle générateur des données est~:
\[
  y_t= X_t\bm\beta + \varepsilon_t
\]
où $X_t$ est un vecteur $1\times K$, $\bm\beta$ un
vecteur $K\times 1$, $\varepsilon_t$ une variable aléatoire iid. On
suppose que les variables $X_t$ sont corrélées avec $\varepsilon_t$,
mais nous disposons de $K$ instrument $Z_t$. Le modèle empirique est~:
\[
  y_t= X_t\bm b + \epsilon_t
\]
\question{1} Donner une expression matricielle de l'estimateur à
variables instrumentales $\hat{\bm b}_{\text{IV}}$. \question{2}
Montrer que cet estimateur est convergent. \question{3} Proposer un
estimateur si le nombre d'instruments est strictment supérieur à $K$.


\end{document}


%%% Local Variables:
%%% mode: latex
%%% TeX-master: t
%%% End:

% 05 68 et 05 69


0948*, 0949*, 0162*, 0163*, 0270*, 0271*, 0377*, 0378*, 0424*, 0425*, 0568*, 0569*
