\synctex=1

\documentclass[10pt]{beamer}

\usepackage[T1]{fontenc}
\usepackage{etex}
\usepackage{fourier-orns}
\usepackage{ccicons}
\usepackage{amssymb}
\usepackage{amstext}
\usepackage{amsbsy}
\usepackage{amsopn}
\usepackage{amscd}
\usepackage{amsxtra}
\usepackage{amsthm}
\usepackage{float}
\usepackage{color, colortbl}
\usepackage{mathrsfs}
\usepackage{bm}
\usepackage[nice]{nicefrac}
\usepackage{setspace}
\usepackage{ragged2e}
\usepackage{listings}
\usepackage{algorithms/algorithm}
\usepackage{algorithms/algorithmic}
\usepackage{tikz,pgfplots,pgfplotstable}
\pgfplotsset{compat=1.18}%newest}
\usetikzlibrary{patterns, arrows, decorations.pathreplacing, decorations.markings, calc}
\pgfplotsset{plot coordinates/math parser=false}
\usetikzlibrary{external}
\tikzexternalize[prefix=figures/]
\newlength\figureheight
\newlength\figurewidth
\usepackage{cancel}
\usepackage{tikz-qtree}
\usepackage{dcolumn}
\usepackage{adjustbox}
\usepackage{environ}
\usepackage[cal=boondox]{mathalfa}
\usepackage{manfnt}
\usepackage{hyperref}
\hypersetup{
  colorlinks=true,
  linkcolor=blue,
  filecolor=black,
  urlcolor=blue,
}
\usepackage{venndiagram}
\usepackage{subcaption}
\usepackage{centernot}

\usepackage[backend=biber,style=bwl-FU,natbib=true,doi=false,isbn=false,url=false,eprint=false]{biblatex}%bwl-FU
\addbibresource{econometrics.bib}

\makeatletter
\@ifclassloaded{beamer}{
\usefonttheme[onlymath]{serif}
\uselanguage{French}
\languagepath{French}
% Git hash
\usepackage{xstring}
\usepackage{catchfile}
\immediate\write18{git rev-parse HEAD > git.hash}
\CatchFileDef{\HEAD}{git.hash}{\endlinechar=-1}
\newcommand{\gitrevision}{\StrLeft{\HEAD}{7}}
}{}
\makeatother

\newcommand{\trace}{\mathrm{tr}}
\newcommand{\vect}{\mathrm{vec}}
\newcommand{\tracarg}[1]{\mathrm{tr}\left\{#1\right\}}
\newcommand{\vectarg}[1]{\mathrm{vec}\left(#1\right)}
\newcommand{\vecth}[1]{\mathrm{vech}\left(#1\right)}
\newcommand{\iid}[2]{\mathrm{iid}\left(#1,#2\right)}
\newcommand{\normal}[2]{\mathcal N\left(#1,#2\right)}
\newcommand{\sample}{\mathcal Y_T}
\newcommand{\samplet}[1]{\mathcal Y_{#1}}
\newcommand{\slidetitle}[1]{\fancyhead[L]{\textsc{#1}}}

\newcommand{\R}{{\mathbb R}}
\newcommand{\C}{{\mathbb C}}
\newcommand{\N}{{\mathbb N}}
\newcommand{\Z}{{\mathbb Z}}
\newcommand{\binomial}[2]{\begin{pmatrix} #1 \\ #2 \end{pmatrix}}
\newcommand{\bigO}[1]{\mathcal O \left(#1\right)}
\newcommand{\red}{\color{red}}
\newcommand{\blue}{\color{blue}}
\newcommand{\plim}{\xrightarrow[T\rightarrow\infty]{\text{proba}}}% {\overset{\text{proba}}{\underset{T\rightarrow\infty}{\longrightarrow}}}
\newcommand{\epsvar}{\sigma_{\varepsilon}^2}


\newcommand\gauss[2]{1/(#2*sqrt(2*pi))*exp(-((x-#1)^2)/(2*#2^2))} % Gaussian probability density function.

\renewcommand{\qedsymbol}{C.Q.F.D.}

\newcolumntype{d}{D{.}{.}{-1}}
\definecolor{gray}{gray}{0.9}
\newcolumntype{g}{>{\columncolor{gray}}c}


\makeatletter
\@ifclassloaded{beamer}{\setbeamertemplate{theorems}[numbered]{}}{}
\makeatother

\theoremstyle{plain}

\makeatletter
\@ifclassloaded{beamer}{
\setbeamertemplate{footline}{
  {\hfill\vspace*{1pt}\href{http://creativecommons.org/licenses/by-sa/3.0/legalcode}{\ccbysa}\hspace{.1cm}
    \href{https://github.com/stepan-a/econometrics/blob/\HEAD/cours/chapitre-3.tex}{\gitrevision}\enspace--\enspace\today\enspace
  }}

\makeatother


\setbeamertemplate{navigation symbols}{}
\setbeamertemplate{blocks}[rounded][shadow=true]
\setbeamertemplate{caption}[numbered]

\NewEnviron{notes}{\justifying\footnotesize\begin{spacing}{1.0}\BODY\vfill\pagebreak\end{spacing}}

\newenvironment{exercise}[1]
{\bgroup \small\begin{block}{Ex. #1}}
  {\end{block}\egroup}

\newenvironment{rem}[1]
{\bgroup \small\begin{block}{Remarque. #1}}
  {\end{block}\egroup}

\newenvironment{defn}[1]
{\bgroup \small\begin{block}{Définition. #1}}
  {\end{block}\egroup}

\newenvironment{exemple}[1]
{\bgroup \small\begin{block}{Exemple. #1}}
  {\end{block}\egroup}
}{}

\newcommand{\dnote}[1]{%
    \noindent % I guess this is intended...
    \begin{tabular}{@{}m{0.13\textwidth}@{}m{0.87\textwidth}@{}}%
        \huge\textdbend &#1%
    \end{tabular}%
    \par % ... and this too.
}

\newtheorem{thm}{Théorème}
\newtheorem{prop}{Proposition}
\newtheorem{cor}{Corollaire}

%\usepgfplotslibrary{external}
%\tikzexternalize

\begin{document}

\title{Économétrie\\\small{Régresseurs non déterministes}}
\author[S. Adjemian]{Stéphane Adjemian}
\institute{\texttt{stephane.adjemian@univ-lemans.fr}}
\date{Septembre 2025}

\begin{frame}
  \titlepage{}
\end{frame}


\begin{frame}
  \frametitle{Régresseurs aléatoires}

  \begin{itemize}

  \item Dans les chapitres \href{https://le-mans.adjemian.eu/econometrics/chapitre-1.pdf}{I} et \href{https://le-mans.adjemian.eu/econometrics/chapitre-2.pdf}{II}, nous avons supposé que les variables explicatives, $X$, sont détermnistes.\newline

  \item Cette hypothèse n'est généralement pas raisonnable.\newline

  \item \textsc{Exemple} Si le modèle est~: $y_t = \rho y_{t-1} + \varepsilon_t$ où $\varepsilon_t$ est une variable aléatoire normale (AR(1)).\newline

  \item Même si la condition initiale, $y_0$, est déterministe, la variable $y$ est clairement aléatoire (à gauche, comme d'habitude, mais aussi à droite).\newline 

  \item Quelles sont les conséqences pour l'inférence~?\newline

  \end{itemize}

\end{frame}


\begin{frame}
  \frametitle{Erreurs de mesure}

  \begin{itemize}

  \item Supposons que le DGP soit $\mathbf y = X^{\star}\beta + \varepsilon$, où les conditions idéales sont vérifiées et où la matrice des régresseurs $X^{\star}$ est déterministe.\newline

  \item Supposons que les variables explicatives soient mesurées avec des erreurs, on observe seulement $X = X^{\star} + \eta$, où $\eta$ est une variable aléatoire (centrée).\newline

  \item Les variables explicatives sont observées à un aléa près $\Rightarrow$ Les variables explicatives, $X$, considérées par l'économètre sont aléatoires.\newline

  \item Est-il possible d'obtenir une estimation sans biais de $\beta$ en régressant $\mathbf y$ sur $X$~? Convergente~? Quelles sont les propriétés de l'estimateur des MCO~? \newline

  \item Ici la variable explicative est aléatoire, on verra que cela biaise l'estimation car elle est corrélée avec l'erreur du modèle.
    
  \end{itemize}

\end{frame}


\begin{frame}
  \frametitle{Double causalité}

  \begin{itemize}

  \item Supposons que l'on s'intéresse à l'effet du revenu sur la santé dans une population. On considère le modèle~:
    \[
      \text{Santé}_i = \beta_0 + \beta_1 \text{Revenu}_i + u_i
    \]

  \item Les individus les plus riches mangent mieux et vivent dans de meilleurs environnements $\rightarrow$ $\beta_1>0$\newline

  \item Mais une meilleure santé accroît la productivité des individus, on peut donc lire la causalité dans l'autre sens, avec par exemple un modèle de la forme~:
    \[
      \text{Revenu}_i = \alpha_0 + \alpha_1 \text{Santé}_i + v_i
    \]
    avec $\alpha_1>0$.

  \item Cette double causalité induit une corrélation entre le revenu de l'individu $i$ et $u_i$. $\Rightarrow$  On verra que cela biaise l'estimation de $\beta_1$ par les MCO dans le premier modèle.
    
  \end{itemize}

\end{frame}


\begin{notes}

  \begin{itemize}

  \item En substituant la seconde équation dans la première (on élimine le revenu), on obtient~:
    \[
      \text{Santé}_i = \beta_0 + \beta_1\left( \alpha_0+\alpha_1\text{Santé}_i+v_i\right)+u_i
    \]
\[
      \Leftrightarrow \text{Santé}_i = \frac{\beta_0+\beta_1\alpha_0}{1-\beta_1\alpha_1} + \frac{\beta_1 v_i}{1-\beta_1\alpha_1} + \frac{u_i}{1-\beta_1\alpha_1}
    \]    
  \item En substituant dans la seconde equation~:
    \[
      \text{Revenu}_i = \alpha_0 + \alpha_1\frac{\beta_0+\beta_1\alpha_0}{1-\beta_1\alpha_1} + \left( 1 + \frac{\alpha_1\beta_1}{1-\alpha_1\beta_1}\right)v_i  + \frac{\alpha_1u_i}{1-\alpha_1\beta_1}
    \]
\[
      \Leftrightarrow\text{Revenu}_i = \alpha_0 + \alpha_1\frac{\beta_0+\beta_1\alpha_0}{1-\beta_1\alpha_1}  + \frac{v_i}{1-\alpha_1\beta_1}  + \frac{\alpha_1u_i}{1-\alpha_1\beta_1}
    \]

  \item La corrélation entre revenu et le terme d'erreur dans le premier modèle est donc non nulle:
    \[
      \textrm{corr}\left( \text{Revenu}_i, u_i \right) = \frac{\alpha_1\sigma_u^2}{1-\alpha_1\beta_1}
    \]
    en supposant que $u_i$ et $v_i$ sont non corrélés.

  \item Clairement les deux variables considérée ici (le revenu et la
    santé) sont aléatoires. C'est la corrélation, dans le premier
    modèle, entre le revenu et le terme d'erreur qui va biaiser
    l'estimateur de $\beta_1$.\newline

  \item Si une personne est en meilleure santé pour des raisons non
    observées (captées par $u_i$), cette meilleure santé accroît aussi
    son revenu, le revenu observé est donc corrélé au terme d’erreur.

  \end{itemize}

\end{notes}


\begin{frame}{Le modèle de la Nature}

  \[ \mathbf y = X\beta + \varepsilon \]

\begin{itemize}

\item $\varepsilon \sim \mathcal N(0, \sigma_{\varepsilon}^2 I_T)$\newline

\item $X$ est une matrice aléatoire $T\times K$\newline

\item $\frac{X'X}{T} \plim  Q $ est une finie et de plein rang.\newline

\item Les régresseurs sont linéairement indépendants avec probabilité 1.\newline

\item Les colonnes de $X$ ne sont pas nécessairement toutes aléatoire (constante).\newline

\item \textbf{Problème:} Le modèle est incomplet puisque nous ne disons rien de la loi de $X$.\newline
  
\end{itemize}

\end{frame}

\begin{frame}{\textbf{\textsc{Cas 1:}} $X$ et $\varepsilon$ indépendants}

\begin{itemize}

\item La distribution de $\varepsilon$ conditionnelle à $X$ est identique à sa distribution marginale.\newline

\item $\varepsilon | X \sim  \mathcal N(0, \sigma_{\varepsilon}^2 I_T)$\newline

\item Tous les résultats du Chapitre 1 sont valides \textbf{conditionnellement à $X$}.\newline

\item L'estimateur MCO conserve ses propriétés désirables.\newline

\item Les tests usuels restent valides (conditionnellement à $X$)\newline

\end{itemize}

\vspace{0.3cm}

\textbf{Intuition:} En conditionnant sur $X$, on le traite les variables explicatives comme non-stochastiques. L'indépendance garantit que la valeur particulière de $X$ soit sans conséquence.

\end{frame}

\begin{frame}{Limites des énoncés conditionnels}

  \begin{itemize}

  \item Si $X$ et $\varepsilon$ sont tous deux stochastiques, l'utilité d'énoncés conditionnels sur $X$ est limitée.\newline

  \item Pour faire des inférences \textbf{non conditionnelles}, il faut connaître la distribution de $X$(difficile).\newline

  \item Si $x_{ti}$ sont iid et si $\mathbb E[x_{ti}\varepsilon_t]$ existe (nul), alors la loi des grands nombres s'applique~:
    \[
      \frac{1}{T}\sum_{t=1}^{T} x_{ti}\varepsilon_t \plim 0 \quad (i = 1, 2, \ldots, K)
    \]    
    Cela assure la convergence de l'estimateur des MCO.\newline
    
\end{itemize}

\dnote{Les $x_{ti}$ peuvent ne pas être iid, et $\mathbb E[x_{ti}\varepsilon_t]$ peut ne pas exister~! On ne peut alors rien dire de la convergence de l'estimateur des MCO.}
\end{frame}

\begin{frame}{Distribution asymptotique}
\textbf{Problème:}
\begin{itemize}
    \item On ne peut pas garantir que $\hat{\beta}$ sera asymptotiquement normal
    \item Les théorèmes de limite centrale ne s'appliquent pas directement car $\hat{\beta}$ est une combinaison \textbf{non-linéaire} de $X$ et $\varepsilon$
    \item La non-linéarité vient de l'inversion de $X'X$
\end{itemize}

\vspace{0.3cm}
\textbf{Sans information sur la distribution de $X$:}
\begin{itemize}
    \item Tests d'hypothèses valides impossibles, même asymptotiquement
\end{itemize}
\end{frame}

\begin{frame}{Cas 2: Corrélation entre $X$ et $\varepsilon$}


\[
  \text{plim} \, \hat{\beta} = \beta + \text{plim} \frac{X'\varepsilon}{T} \neq \beta
\]

\bigskip\bigskip

$\Rightarrow$ \textbf{$\hat{\beta}$ est inconsistant}

\end{frame}

% ============================================
% Section 3.3: Instrumental Variables
% ============================================

\section{3.3 Variables Instrumentales}

\begin{frame}{3.3 Variables Instrumentales: Le problème}
\textbf{Contexte:}
\[ y = X\beta + \varepsilon \]

\textbf{Problème:}
\[ \text{plim} \frac{1}{T}X'\varepsilon \neq 0 \]

$\Rightarrow$ L'estimateur MCO $\hat{\beta} = (X'X)^{-1}X'y$ est \textbf{inconsistant}

\vspace{0.5cm}
\textbf{Solution:} Trouver des \textbf{instruments} $Z$ qui remplacent $X$ dans la construction de l'estimateur
\end{frame}

\begin{frame}{Théorème principal}
\begin{thm}
Supposons qu'il existe un ensemble de variables $Z$ tel que:
\begin{enumerate}
    \item $Q_{ZX} = \text{plim} \frac{1}{T}Z'X$ est finie et non-singulière
    \item $\frac{Z'\varepsilon}{\sqrt{T}} \xrightarrow{d} N(0, \Psi)$
\end{enumerate}

Alors l'estimateur
\[ \tilde{\beta} = (Z'X)^{-1}Z'y \]
est consistant, et la distribution asymptotique de $\sqrt{T}(\tilde{\beta} - \beta)$ est:
\[ N\left(0, Q_{ZX}^{-1}\Psi(Q_{ZX}')^{-1}\right) \]
\end{thm}
\end{frame}

\begin{frame}{Preuve: Consistance}
\textbf{Preuve de la consistance:}

On a:
\[ \tilde{\beta} = (Z'X)^{-1}Z'y = (Z'X)^{-1}Z'(X\beta + \varepsilon) \]
\[ = \beta + (Z'X)^{-1}Z'\varepsilon \]

Donc:
\[ \text{plim} \, \tilde{\beta} = \beta + Q_{ZX}^{-1} \cdot \text{plim} \frac{Z'\varepsilon}{T} \]

Puisque $\frac{Z'\varepsilon}{\sqrt{T}}$ a une distribution asymptotique bien définie, nécessairement:
\[ \text{plim} \frac{Z'\varepsilon}{T} = 0 \]

D'où: $\text{plim} \, \tilde{\beta} = \beta$ \hfill $\square$
\end{frame}

\begin{frame}{Preuve: Distribution asymptotique}
\textbf{Distribution asymptotique:}

On a:
\[ \sqrt{T}(\tilde{\beta} - \beta) = \left(\frac{Z'X}{T}\right)^{-1} \frac{Z'\varepsilon}{\sqrt{T}} \]

Puisque:
\begin{itemize}
    \item $\frac{Z'\varepsilon}{\sqrt{T}} \xrightarrow{d} N(0, \Psi)$
    \item $\frac{Z'X}{T} \xrightarrow{p} Q_{ZX}$
\end{itemize}

Par le théorème de Slutsky:
\[ \sqrt{T}(\tilde{\beta} - \beta) \xrightarrow{d} N\left(0, Q_{ZX}^{-1}\Psi(Q_{ZX}')^{-1}\right) \]
\hfill $\square$
\end{frame}

\begin{frame}{Définitions et remarques}
\begin{defn}{}
L'estimateur $\tilde{\beta} = (Z'X)^{-1}Z'y$ est appelé \textbf{estimateur par variables instrumentales} (IV) de $\beta$. La matrice $Z$ est l'ensemble des \textbf{instruments} pour $X$.
\end{defn}

\vspace{0.3cm}
\begin{rem}{}
Le cas typique est celui où:
\[ \frac{Z'\varepsilon}{\sqrt{T}} \xrightarrow{d} N(0, \sigma^2 Q_{ZZ}) \]
où $Q_{ZZ} = \text{plim} \frac{1}{T}Z'Z$
\end{rem}
\end{frame}

\begin{frame}{Exemple: Offre et demande (simultanéité)}
\textbf{Modèle de marché:}

Demande (consommateurs):
\[ Q_t^D = \alpha_0 + \alpha_1 P_t + \alpha_2 R_t + u_t \]
où $R_t$ est le revenu des consommateurs, $\alpha_1 < 0$

Offre (producteurs):
\[ Q_t^S = \beta_0 + \beta_1 P_t + \beta_2 W_t + v_t \]
où $W_t$ est le coût des inputs, $\beta_1 > 0$

Équilibre:
\[ Q_t^D = Q_t^S = Q_t \]

\textbf{Problème:} Le prix $P_t$ est \textbf{endogène} (déterminé simultanément avec $Q_t$)
\end{frame}

\begin{frame}{Exemple (suite): Prix d'équilibre}
\textbf{Résolution du système:}

À l'équilibre, $Q_t^D = Q_t^S$:
\[ \alpha_0 + \alpha_1 P_t + \alpha_2 R_t + u_t = \beta_0 + \beta_1 P_t + \beta_2 W_t + v_t \]

Le prix d'équilibre est:
\[ P_t = \frac{(\beta_0 - \alpha_0) + \beta_2 W_t - \alpha_2 R_t + (v_t - u_t)}{\alpha_1 - \beta_1} \]

\textbf{Observation cruciale:}
\[ \mathbb{E}[P_t \cdot u_t] = \mathbb{E}\left[\frac{v_t - u_t}{\alpha_1 - \beta_1} \cdot u_t\right] = \frac{-\sigma_u^2}{\alpha_1 - \beta_1} \neq 0 \]

$\Rightarrow$ Le prix est corrélé avec l'erreur de demande!

$\Rightarrow$ MCO sur l'équation de demande est \textbf{inconsistant}
\end{frame}

\begin{frame}{Exemple (suite): Estimation par VI}
\textbf{Objectif:} Estimer l'équation de demande
\[ Q_t = \alpha_0 + \alpha_1 P_t + \alpha_2 R_t + u_t \]

\textbf{Instruments pour $P_t$:}
\[ Z_t = (1, R_t, W_t)' \]

\textbf{Justification:}
\begin{itemize}
    \item $W_t$ (coût des inputs) affecte l'offre mais pas directement la demande
    \item $W_t$ est corrélé avec $P_t$ (via l'équation d'offre)
    \item $W_t$ n'est pas corrélé avec $u_t$ (erreur de demande)
    \item $R_t$ (revenu) peut être utilisé comme instrument pour lui-même
\end{itemize}
\end{frame}

\begin{frame}{Exemple (suite): Vérification des conditions}
\textbf{Conditions du théorème IV:}

\begin{enumerate}
    \item \textbf{Pertinence:} $Q_{ZX} = \text{plim} \frac{1}{T}Z'X$ finie et non-singulière
    \begin{itemize}
        \item $X = (1, P_t, R_t)$, $Z = (1, R_t, W_t)$
        \item $W_t$ est corrélé avec $P_t$ via l'équation d'offre
        \item Si $\text{Var}(W_t) > 0$ et $\beta_2 \neq 0$, alors $Q_{ZX}$ est non-singulière
    \end{itemize}

    \item \textbf{Exogénéité:} $\frac{Z'u}{\sqrt{T}} \xrightarrow{d} N(0, \Psi)$
    \begin{itemize}
        \item Supposons que $(R_t, W_t)$ sont stationnaires avec moments finis
        \item Supposons que $u_t$ est iid $N(0, \sigma_u^2)$ et indépendant de $(R_s, W_s)$ pour tout $t,s$
        \item Alors par un TCL pour processus stochastiques:
        \[ \frac{1}{\sqrt{T}}\sum_{t=1}^{T}Z_t u_t \xrightarrow{d} N(0, \sigma_u^2 Q_{ZZ}) \]
        où $Q_{ZZ} = \text{plim} \frac{1}{T}\sum_{t=1}^{T} Z_t Z_t'$
    \end{itemize}
\end{enumerate}

$\Rightarrow$ L'estimateur IV $\tilde{\alpha} = (Z'X)^{-1}Z'Q$ est \textbf{consistant}
\end{frame}

\begin{frame}{Exemple (suite): Pourquoi ces conditions sur $Z$?}
\textbf{Question:} Pourquoi avons-nous besoin d'hypothèses sur $(R_t, W_t)$ pour appliquer le TCL?

\vspace{0.3cm}
\textbf{Réponse:} Le TCL s'applique à la somme $\frac{1}{\sqrt{T}}\sum_{t=1}^{T}Z_t u_t$

\begin{itemize}
    \item Même si $u_t$ est iid, le produit $Z_t u_t$ est \textbf{stochastique}\newline

    \item Pour que le TCL s'applique, il faut que:\newline
    \begin{enumerate}
        \item $\mathbb{E}[Z_t u_t] = 0$ (exogénéité des instruments)\newline
        \item $\text{Var}(Z_t u_t) < \infty$ (nécessite moments finis de $Z_t$)\newline
        \item La LGN s'applique à $\frac{1}{T}\sum Z_t Z_t' u_t^2$ (stationnarité)
    \end{enumerate}

\end{itemize}
\end{frame}

\begin{frame}{Exemple (suite): Ce qui peut mal tourner}
\textbf{Ce qui peut mal tourner sans conditions sur $Z$:}

\begin{enumerate}
    \item \textbf{Variance infinie:} Si $Z_t$ a des queues trop épaisses
    \begin{itemize}
        \item $\text{Var}(Z_t u_t)$ peut ne pas exister
        \item Le TCL ne s'applique pas
    \end{itemize}

    \item \textbf{Dépendance temporelle:} Si $Z_t$ est très persistant
    \begin{itemize}
        \item Les $Z_t u_t$ ne sont pas "suffisamment indépendants"
        \item La normalisation par $\sqrt{T}$ peut être incorrecte
    \end{itemize}

    \item \textbf{Non-stationnarité:} Si $\text{Var}(Z_t)$ croît avec $t$
    \begin{itemize}
        \item $Q_{ZZ} = \text{plim} \frac{1}{T}\sum Z_t Z_t'$ peut ne pas exister
        \item La distribution asymptotique est mal définie
    \end{itemize}
\end{enumerate}

\vspace{0.3cm}
\textbf{Conclusion:} Les conditions sur $Z$ garantissent que la somme $\sum Z_t u_t$ se comporte "régulièrement"
\end{frame}

\begin{frame}{Mise en garde importante}
\textbf{Erreur courante dans la littérature:}

\begin{itemize}
    \item On affirme parfois que si $Q_{ZX}$ est finie et non-singulière et si:
    \[ \text{plim} \frac{1}{T}Z'\varepsilon = 0 \]
    alors $\tilde{\beta}$ est consistant et $\sqrt{T}(\tilde{\beta} - \beta) \xrightarrow{d} N(0, \sigma^2 Q_{ZZ}^{-1}Q_{ZX}(Q_{ZX}')^{-1})$
\end{itemize}

\vspace{0.3cm}
\textbf{FAUX!}

\begin{itemize}
    \item La consistance est vraie
    \item Mais l'énoncé sur la distribution asymptotique est \textbf{faux}
    \item Raison: $\text{plim} \frac{1}{T}Z'\varepsilon = 0$ n'implique pas que $\frac{Z'\varepsilon}{\sqrt{T}} \xrightarrow{d} N(0, \sigma^2 Q_{ZZ})$
\end{itemize}
\end{frame}


\begin{frame}{Difficulté de vérification}
\textbf{Problème pratique:}

Il est clairement plus facile de vérifier:
\[ \text{plim} \frac{1}{T}Z'\varepsilon = 0 \]

que de vérifier:
\[ \frac{Z'\varepsilon}{\sqrt{T}} \text{ a une distribution asymptotique bien définie} \]

\vspace{0.3cm}
\textbf{Question naturelle:} Existe-t-il des conditions suffisantes facilement vérifiables?

\textbf{Réponse:} Si $Z$ est non-stochastique et $Q_{ZZ} = \text{plim} \frac{1}{T}Z'Z$ est finie, c'est suffisant.

\textbf{Mais:} Si $Z$ est stochastique, c'est compliqué. Les conditions suffisantes qui existent sont trop exigeantes pour être vraiment utiles.
\end{frame}

\begin{frame}{Conditions fortes (mais insuffisantes)}
\textbf{Tentative de condition:} Les observations $Z_t$ sont iid et indépendantes de toutes les observations sur $\varepsilon$

\vspace{0.3cm}
\textbf{Mais même cela ne suffit pas!}

\textbf{Contre-exemple:} Reprenons l'exemple de la Section 3.1 où:
\[ Z_t = \frac{1}{X_t} \]
avec $X_t$ iid $N(0, \sigma_X^2)$ et indépendant de $\varepsilon$.

\begin{itemize}
    \item Les $Z_t$ sont aussi iid et indépendants de $\varepsilon$
    \item \textbf{Mais:} $Z_t\varepsilon_t$ a une distribution de Cauchy (dont la moyenne n'existe pas)
    \item Donc $\frac{1}{T}\sum_{t=1}^{T} Z_t\varepsilon_t$ aussi
    \item On n'a même pas $\text{plim} \frac{1}{T}Z'\varepsilon = 0$
    \item Encore moins une distribution asymptotique bien définie pour $\frac{Z'\varepsilon}{\sqrt{T}}$
\end{itemize}
\end{frame}

% ============================================
% Section 3.4: Errors in Variables
% ============================================

\section{3.4 Erreurs de Mesure}

\begin{frame}{3.4 Erreurs de Mesure: Le modèle}
\textbf{Modèle de régression linéaire simple:}
\[ y_t = \alpha + \beta x_t + \varepsilon_t \]

\textbf{Problème:} $y$ et $x$ sont observés avec erreur

\textbf{Variables observées:}
\begin{align*}
y_t^* &= y_t + v_t = \alpha + \beta x_t + (\varepsilon_t + v_t) \\
x_t^* &= x_t + u_t
\end{align*}

\textbf{Forme alternative:}
\[ y_t^* = \alpha + \beta x_t^* + (\varepsilon_t + v_t - \beta u_t) \]

\textbf{Observation:} La perturbation composée dépend de $x_t^*$ via $u_t$!
\end{frame}

\begin{frame}{Théorème d'inconsistance}
\begin{thm}
Considérons le modèle ci-dessus, où $\varepsilon_t$, $v_t$, $u_t$ et $x_t$ sont tous indépendants entre eux, et sont de plus iid comme $N(0,\sigma_\varepsilon^2)$, $N(0,\sigma_v^2)$, $N(0,\sigma_u^2)$ et $N(\mu,\sigma_x^2)$ respectivement.

Alors l'estimateur MCO $\hat{\beta}$ de $\beta$ est \textbf{inconsistant} tant que $\sigma_u^2 \neq 0$, et:
\[ \text{plim} \, \hat{\beta} = \beta \frac{\sigma_x^2}{\sigma_x^2 + \sigma_u^2} \]
\end{thm}

\textbf{Biais:} $\text{plim} \, \hat{\beta} < \beta$ (en valeur absolue)

C'est un \textbf{biais d'atténuation} vers zéro.
\end{frame}

\begin{frame}{Preuve du théorème}
\textbf{Preuve:}

L'estimateur MCO de $\beta$ est:
\[ \hat{\beta} = \frac{\sum_t(x_t^* - \bar{x}^*)(y_t^* - \bar{y}^*)}{\sum_t(x_t^* - \bar{x}^*)^2} = \frac{(1/T)\sum_t(x_t^* - \bar{x}^*)(\varepsilon_t + v_t - \beta u_t)}{(1/T)\sum_t(x_t^* - \bar{x}^*)^2} \]

Puisque $x_t^* = x_t + u_t$, sous les hypothèses:
\begin{align*}
\text{plim} \frac{1}{T}\sum_t(x_t^* - \bar{x}^*)(\varepsilon_t + v_t - \beta u_t) &= -\beta\sigma_u^2 \\
\text{plim} \frac{1}{T}\sum_t(x_t^* - \bar{x}^*)^2 &= \sigma_x^2 + \sigma_u^2
\end{align*}

D'où:
\[ \text{plim} \, \hat{\beta} = \beta \frac{-\sigma_u^2}{\sigma_x^2 + \sigma_u^2} + \beta = \beta \frac{\sigma_x^2}{\sigma_x^2 + \sigma_u^2} \]
\hfill $\square$
\end{frame}

\begin{frame}{La mesure de $y$ ne pose pas de problème}
\begin{rem}{}
C'est l'\textbf{erreur de mesure sur $x$} qui cause le problème.

L'erreur $v_t$ sur $y$ est indistinguable de la perturbation habituelle $\varepsilon_t$.

Pour le reste de cette section, on incorporera donc $\varepsilon_t$ dans $v_t$.
\end{rem}

\vspace{0.5cm}
\textbf{Simplification:}

On considère désormais:
\begin{align*}
y_t^* &= y_t + v_t = \alpha + \beta x_t + v_t \\
x_t^* &= x_t + u_t
\end{align*}
\end{frame}


\begin{frame}{Lien avec les variables instrumentales}
\textbf{Question:} Peut-on utiliser un estimateur IV pour corriger l'erreur de mesure?

\vspace{0.3cm}
\textbf{Réponse:} Oui! L'erreur de mesure est un cas particulier d'endogénéité.

\vspace{0.3cm}
\textbf{Rappel du problème:}
\begin{align*}
y_t &= \alpha + \beta x_t + v_t \\
x_t^* &= x_t + u_t \\
\Rightarrow y_t &= \alpha + \beta x_t^* + \underbrace{(v_t - \beta u_t)}_{\text{erreur composée}}
\end{align*}

Corrélation: $\text{Cov}(x_t^*, v_t - \beta u_t) = -\beta\sigma_u^2 \neq 0$

\vspace{0.3cm}
$\Rightarrow$ Problème d'endogénéité classique, où $x_t^*$ est corrélé avec l'erreur!
\end{frame}

\begin{frame}{Variables instrumentales pour erreurs de mesure}
\textbf{Solution IV:} Trouver un instrument $Z_t$ tel que:

\begin{enumerate}
    \item \textbf{Pertinence:} $Z_t$ corrélé avec $x_t$ (ou $x_t^*$)
    \[ \text{Cov}(Z_t, x_t) \neq 0 \]

    \item \textbf{Exogénéité:} $Z_t$ non corrélé avec les erreurs $u_t$ et $v_t$
    \[ \text{Cov}(Z_t, u_t) = 0 \quad \text{et} \quad \text{Cov}(Z_t, v_t) = 0 \]
\end{enumerate}

\vspace{0.3cm}
\textbf{Exemples d'instruments possibles:}

\begin{itemize}
    \item \textbf{Mesures multiples:} Si on a deux mesures indépendantes de $x_t$
    \begin{itemize}
        \item $x_t^{(1)} = x_t + u_t^{(1)}$ et $x_t^{(2)} = x_t + u_t^{(2)}$
        \item On peut utiliser $x_t^{(2)}$ comme instrument pour $x_t^{(1)}$
        \item Si $u_t^{(1)} \perp u_t^{(2)}$, alors $\text{Cov}(x_t^{(2)}, u_t^{(1)}) = 0$
    \end{itemize}

    \item \textbf{Variables liées:} Une variable $Z_t$ qui affecte $x_t$ mais pas directement $y_t$
\end{itemize}
\end{frame}

\begin{frame}{Exemple: Mesures multiples}
\textbf{Exemple concret:} Mesure du revenu avec erreur

\begin{itemize}
    \item Modèle: $\text{Consommation}_t = \alpha + \beta \text{Revenu}_t + v_t$

    \item Problème: Le revenu est mesuré avec erreur
    \begin{itemize}
        \item Déclaration fiscale: $\text{Revenu}_t^{(1)} = \text{Revenu}_t + u_t^{(1)}$
        \item Enquête auprès des ménages: $\text{Revenu}_t^{(2)} = \text{Revenu}_t + u_t^{(2)}$
    \end{itemize}

    \item Si $u_t^{(1)}$ et $u_t^{(2)}$ sont des erreurs de mesure indépendantes:
    \begin{itemize}
        \item Utiliser $\text{Revenu}_t^{(2)}$ comme instrument pour $\text{Revenu}_t^{(1)}$
        \item Ou vice-versa
    \end{itemize}
\end{itemize}

\vspace{0.3cm}
\textbf{Estimateur IV:}
\[ \tilde{\beta} = \frac{\sum_t (x_t^{(2)} - \bar{x}^{(2)})(y_t - \bar{y})}{\sum_t (x_t^{(2)} - \bar{x}^{(2)})(x_t^{(1)} - \bar{x}^{(1)})} \]

Cet estimateur est \textbf{consistant} pour $\beta$!
\end{frame}



\begin{notes}

  \begin{center}
    \begin{tabular}{c}
      \\
      \Huge{\textsc{Références}}\\
      \\
    \end{tabular}
  \end{center}

  \bigskip

  \nocite{Green2017}

  \nocite{Schmidt1976}

  \printbibliography

\end{notes}


\end{document}


% Local Variables:
% ispell-check-comments: exclusive
% ispell-local-dictionary: "french"
% TeX-master: t
% End:
