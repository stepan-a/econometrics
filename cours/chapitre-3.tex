\synctex=1

\documentclass[10pt]{beamer}

\usepackage[T1]{fontenc}
\usepackage{etex}
\usepackage{fourier-orns}
\usepackage{ccicons}
\usepackage{amssymb}
\usepackage{amstext}
\usepackage{amsbsy}
\usepackage{amsopn}
\usepackage{amscd}
\usepackage{amsxtra}
\usepackage{amsthm}
\usepackage{float}
\usepackage{color, colortbl}
\usepackage{mathrsfs}
\usepackage{bm}
\usepackage[nice]{nicefrac}
\usepackage{setspace}
\usepackage{ragged2e}
\usepackage{listings}
\usepackage{algorithms/algorithm}
\usepackage{algorithms/algorithmic}
\usepackage{tikz,pgfplots,pgfplotstable}
\pgfplotsset{compat=1.18}%newest}
\usetikzlibrary{patterns, arrows, decorations.pathreplacing, decorations.markings, calc}
\pgfplotsset{plot coordinates/math parser=false}
\usetikzlibrary{external}
\tikzexternalize[prefix=figures/]
\newlength\figureheight
\newlength\figurewidth
\usepackage{cancel}
\usepackage{tikz-qtree}
\usepackage{dcolumn}
\usepackage{adjustbox}
\usepackage{environ}
\usepackage[cal=boondox]{mathalfa}
\usepackage{manfnt}
\usepackage{hyperref}
\hypersetup{
  colorlinks=true,
  linkcolor=blue,
  filecolor=black,
  urlcolor=blue,
}
\usepackage{venndiagram}
\usepackage{subcaption}
\usepackage{centernot}

\usepackage[backend=biber,style=bwl-FU,natbib=true,doi=false,isbn=false,url=false,eprint=false]{biblatex}%bwl-FU
\addbibresource{econometrics.bib}

\makeatletter
\@ifclassloaded{beamer}{
\usefonttheme[onlymath]{serif}
\uselanguage{French}
\languagepath{French}
% Git hash
\usepackage{xstring}
\usepackage{catchfile}
\immediate\write18{git rev-parse HEAD > git.hash}
\CatchFileDef{\HEAD}{git.hash}{\endlinechar=-1}
\newcommand{\gitrevision}{\StrLeft{\HEAD}{7}}
}{}
\makeatother

\newcommand{\trace}{\mathrm{tr}}
\newcommand{\vect}{\mathrm{vec}}
\newcommand{\tracarg}[1]{\mathrm{tr}\left\{#1\right\}}
\newcommand{\vectarg}[1]{\mathrm{vec}\left(#1\right)}
\newcommand{\vecth}[1]{\mathrm{vech}\left(#1\right)}
\newcommand{\iid}[2]{\mathrm{iid}\left(#1,#2\right)}
\newcommand{\normal}[2]{\mathcal N\left(#1,#2\right)}
\newcommand{\sample}{\mathcal Y_T}
\newcommand{\samplet}[1]{\mathcal Y_{#1}}
\newcommand{\slidetitle}[1]{\fancyhead[L]{\textsc{#1}}}

\newcommand{\R}{{\mathbb R}}
\newcommand{\C}{{\mathbb C}}
\newcommand{\N}{{\mathbb N}}
\newcommand{\Z}{{\mathbb Z}}
\newcommand{\binomial}[2]{\begin{pmatrix} #1 \\ #2 \end{pmatrix}}
\newcommand{\bigO}[1]{\mathcal O \left(#1\right)}
\newcommand{\red}{\color{red}}
\newcommand{\blue}{\color{blue}}
\newcommand{\plim}{\overset{\text{proba}}{\underset{T\rightarrow\infty}{\longrightarrow}}}
\newcommand{\epsvar}{\sigma_{\varepsilon}^2}


\newcommand\gauss[2]{1/(#2*sqrt(2*pi))*exp(-((x-#1)^2)/(2*#2^2))} % Gaussian probability density function.

\renewcommand{\qedsymbol}{C.Q.F.D.}

\newcolumntype{d}{D{.}{.}{-1}}
\definecolor{gray}{gray}{0.9}
\newcolumntype{g}{>{\columncolor{gray}}c}


\makeatletter
\@ifclassloaded{beamer}{\setbeamertemplate{theorems}[numbered]{}}{}
\makeatother

\theoremstyle{plain}

\makeatletter
\@ifclassloaded{beamer}{
\setbeamertemplate{footline}{
  {\hfill\vspace*{1pt}\href{http://creativecommons.org/licenses/by-sa/3.0/legalcode}{\ccbysa}\hspace{.1cm}
    \href{https://github.com/stepan-a/econometrics/blob/\HEAD/cours/chapitre-3.tex}{\gitrevision}\enspace--\enspace\today\enspace
  }}

\makeatother


\setbeamertemplate{navigation symbols}{}
\setbeamertemplate{blocks}[rounded][shadow=true]
\setbeamertemplate{caption}[numbered]

\NewEnviron{notes}{\justifying\footnotesize\begin{spacing}{1.0}\BODY\vfill\pagebreak\end{spacing}}

\newenvironment{exercise}[1]
{\bgroup \small\begin{block}{Ex. #1}}
  {\end{block}\egroup}

\newenvironment{defn}[1]
{\bgroup \small\begin{block}{Définition. #1}}
  {\end{block}\egroup}

\newenvironment{exemple}[1]
{\bgroup \small\begin{block}{Exemple. #1}}
  {\end{block}\egroup}
}{}

\newtheorem{prop}{Proposition}
\newtheorem{cor}{Corollaire}

%\usepgfplotslibrary{external}
%\tikzexternalize


\begin{document}

\title{Économétrie\\\small{Régresseurs non déterministes}}
\author[S. Adjemian]{Stéphane Adjemian}
\institute{\texttt{stephane.adjemian@univ-lemans.fr}}
\date{Septembre 2025}

\begin{frame}
  \titlepage{}
\end{frame}


\begin{frame}
  \frametitle{Régresseurs aléatoires}

  \begin{itemize}

  \item Dans les chapitres \href{https://le-mans.adjemian.eu/econometrics/chapitre-1.pdf}{I} et \href{https://le-mans.adjemian.eu/econometrics/chapitre-2.pdf}{II}, nous avons supposé que les variables explicatives, $X$, sont détermnistes.\newline

  \item Cette hypothèse n'est généralement pas raisonnable.\newline

  \item \textsc{Exemple} Si le modèle est~: $y_t = \rho y_{t-1} + \varepsilon_t$ où $\varepsilon_t$ est une variable aléatoire normale (AR(1)).\newline

  \item Même si la condition initiale, $y_0$, est déterministe, la variable $y$ est clairement aléatoire (à gauche, comme d'habitude, mais aussi à droite).\newline 

  \item Quelles sont les conséqences pour l'inférence~?\newline

  \end{itemize}

\end{frame}


\begin{frame}
  \frametitle{Erreurs de mesure}

  \begin{itemize}

  \item Supposons que le DGP soit $\mathbf y = X^{\star}\beta + \varepsilon$, où les conditions idéales sont vérifiées et où la matrice des régresseurs $X^{\star}$ est déterministe.\newline

  \item Supposons que les variables explicatives soient mesurées avec des erreurs, on observe seulement $X = X^{\star} + \eta$, où $\eta$ est une variable aléatoire (centrée).\newline

  \item Les variables explicatives sont observées à un aléa près $\Rightarrow$ Les variables explicatives, $X$, considérées par l'économètre sont aléatoires.\newline

  \item Est-il possible d'obtenir une estimation sans biais de $\beta$ en régressant $\mathbf y$ sur $X$~? Convergente~? Quelles sont les propriétés de l'estimateur des MCO~? \newline

  \item Ici la variable explicative est aléatoire, on verra que cela biaise l'estimation car elle est corrélée avec l'erreur du modèle.
    
  \end{itemize}

\end{frame}


\begin{frame}
  \frametitle{Double causalité}

  \begin{itemize}

  \item Supposons que l'on s'intéresse à l'effet du revenu sur la santé dans une population. On considère le modèle~:
    \[
      \text{Santé}_i = \beta_0 + \beta_1 \text{Revenu}_i + u_i
    \]

  \item Les individus les plus riches mangent mieux et vivent dans de meilleurs environnements $\rightarrow$ $\beta_1>0$\newline

  \item Mais une meilleure santé accroît la productivité des individus, on peut donc lire la causalité dans l'autre sens, avec par exemple un modèle de la forme~:
    \[
      \text{Revenu}_i = \alpha_0 + \alpha_1 \text{Santé}_i + v_i
    \]
    avec $\alpha_1>0$.

  \item Cette double causalité induit une corrélation entre le revenu de l'individu $i$ et $u_i$.

  \item[$\Rightarrow$] On verra que cela biaise l'estimation de $\beta_1$ par les MCO dans le premier modèle.
    
  \end{itemize}

\end{frame}


\begin{notes}

  \begin{itemize}

  \item En substituant la seconde équation dans la première (on élimine le revenu), on obtient~:
    \[
      \text{Santé}_i = \beta_0 + \beta_1\left( \alpha_0+\alpha_1\text{Santé}_i+v_i\right)+u_i
    \]
\[
      \Leftrightarrow \text{Santé}_i = \frac{\beta_0+\beta_1\alpha_0}{1-\beta_1\alpha_1} + \frac{\beta_1 v_i}{1-\beta_1\alpha_1} + \frac{u_i}{1-\beta_1\alpha_1}
    \]    
  \item En substituant dans la seconde equation~:
    \[
      \text{Revenu}_i = \alpha_0 + \alpha_1\frac{\beta_0+\beta_1\alpha_0}{1-\beta_1\alpha_1} + \left( 1 + \frac{\alpha_1\beta_1}{1-\alpha_1\beta_1}\right)v_i  + \frac{\alpha_1u_i}{1-\alpha_1\beta_1}
    \]
\[
      \Leftrightarrow\text{Revenu}_i = \alpha_0 + \alpha_1\frac{\beta_0+\beta_1\alpha_0}{1-\beta_1\alpha_1}  + \frac{v_i}{1-\alpha_1\beta_1}  + \frac{\alpha_1u_i}{1-\alpha_1\beta_1}
    \]

  \item La corrélation entre revenu et le terme d'erreur dans le premier modèle est donc non nulle:
    \[
      \textrm{corr}\left( \text{Revenu}_i, u_i \right) = \frac{\alpha_1\sigma_u^2}{1-\alpha_1\beta_1}
    \]
    en supposant que $u_i$ et $v_i$ sont non corrélés.

  \item Clairement les deux variables considérée ici (le revenu et la
    santé) sont aléatoires. C'est la corrélation, dans le premier
    modèle, entre le revenu et le terme d'erreur qui va biaiser
    l'estimateur de $\beta_1$.\newline

  \item Si une personne est en meilleure santé pour des raisons non
    observées (captées par $u_i$), cette meilleure santé accroît aussi
    son revenu, le revenu observé est donc corrélé au terme d’erreur.

  \end{itemize}

\end{notes}



\begin{notes}

  \begin{center}
    \begin{tabular}{c}
      \\
      \Huge{\textsc{Références}}\\
      \\
    \end{tabular}
  \end{center}

  \bigskip

  \nocite{Green2017}

  \nocite{Schmidt1976}

  \printbibliography

\end{notes}


\end{document}


% Local Variables:
% ispell-check-comments: exclusive
% ispell-local-dictionary: "french"
% TeX-master: t
% End:
