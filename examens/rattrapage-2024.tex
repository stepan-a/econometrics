\documentclass[12pt,a4paper,notitlepage,twocolumn]{article}
\usepackage{amsmath}
\usepackage{amssymb}
\usepackage{amsbsy}
\usepackage{float}
\usepackage[french]{babel}

\usepackage{palatino}

 \usepackage[active]{srcltx}
\usepackage{scrtime}

\newcounter{qnumber}
\setcounter{qnumber}{0}

\newcounter{enumber}
\setcounter{enumber}{0}

\newcommand{\question}{\textbf{(\addtocounter{qnumber}{1}\theqnumber)}\,}
\newcommand{\exercice}{\textbf{\addtocounter{enumber}{1}\textsc{Exercice} \theenumber}.\,}
\setlength{\parindent}{0cm}


\begin{document}

\title{\textsc{Économétrie Approfondie}}
\date{Mardi 17 juin 2025}

\maketitle

\thispagestyle{empty}

\begin{quote}
  \textit{Les réponses non commentées ou insuffisamment détaillées ne seront pas
    considérées. Prenez le temps de faire des phrase.}
\end{quote}

\bigskip


\exercice On suppose que les données sont générées par le modèle suivant~:
\[
  y_i = x_{1,i}\beta_1 + \ldots x_{K,i}\beta_K + \varepsilon_i
\]
où $x_{k,i}$, pour $k=1,\ldots,K$ sont des variables explicatives
déterministes, $\beta_k$, pour $k=1,\ldots,K$, sont des paramètres
réels, $\varepsilon_i$ une variable aléatoire i.i.d. centrée de
variance $\sigma_{\varepsilon}^2$. \question Donner la représentation
matricielle de ce modèle. \question Définir et donner l'expression de
l'estimateur des MCO du vecteur de paramètres $\beta$ (en montrant
comment on arrive à cette expression et en explicitant les hypothèses
nécessaires pour que cet estimateur existe). \question Montrer que cet
estimateur est sans biais. \question Calculer la variance de cet
estimateur. \question Pourquoi l'estimateur des MCO a t-il une
variance~? \question On suppose maintenant que les
erreurs $\varepsilon_i$ ne sont plus i.i.d. (elles sont
hétéroscédastiques et corrélées). Montrer que l'estimateur de MCO est
toujours un estimateur sans biais. \question Montrer comment cela
affecte la variance de l'estimateur des MCO. \question Proposer un
estimateur efficace et calculer sa variance. \newline

\setcounter{qnumber}{0}
\bigskip



\exercice On considère le modèle suivant~:
\[
y_i = \beta_0 + x_i^\star\beta_1 + \varepsilon_i
\]
pour $i=1,\ldots,N$ avec $\beta_0$ et $\beta_1$ des paramètres
réels, $\varepsilon_i$ une variable aléatoire réelle iid d'espérance
nulle et de variance $\sigma_{\varepsilon}^2$, $x_i^\star$ une
variable explicative aléatoire iid de variance $\sigma_{x^*}^2$. Malheureusement, la variable
explicative est observée avec erreur, on n'observe pas $x_i^\star$ mais~:
\[
x_i = x_i^{\star} + \nu_i
\]
avec $\nu_i$ une variable aléaoire iid d'espérance nulle, de
variance $\sigma^2_\nu$ et non corrélée $\varepsilon_j$ ou $x_l^\star$. Le modèle empirique est~:
\[
  y_i = b_0 + x_i b_1 + \epsilon_i
\]
\question Écrire le modèle générateur des données en exprimant $y_i$
en fonction de $x_i$. Une estimation par les MCO peut-elle fournir un
estimateur sans biais de $\beta_1$~? Pourquoi~? \question Écrire
l'expression de $\hat b_1$. \question Calculer la variance de $x$ que
nous noterons $\sigma_x^2$. \question Déterminer le comportement
asymptotique des statistiques
suivantes~:\\ $N^{-1}\sum_{i=1}^N\left( x_i-\bar x \right)^2$,\\ $N^{-1}\sum_{i=1}^N\left( x_i-\bar x \right)\left(\varepsilon_i-\bar\varepsilon  \right)$,\\
$N^{-1}\sum_{i=1}^N\left( x_i^\star-\bar x^\star \right)\left(\nu_i-\bar\nu  \right)$, et\\
$N^{-1}\sum_{i=1}^N\left(\nu_i-\bar\nu  \right)^2$\\
\question Montrer que~:
\[
\text{plim}_{N\rightarrow\infty} \hat b_1 = \beta_1\left( 1 - \frac{\sigma_\nu^2}{\sigma_{x^\star}^2+\sigma_\nu^2} \right)
\]
Interpréter et commenter ce résultat.


\setcounter{qnumber}{0}
\bigskip

\exercice Sur le marché d'un bien la demande et l'offre à la date $t$ sont données par~:
\begin{equation}\label{s1}
  \begin{cases}
    y_t^d &= \alpha + \beta P_t + u_t\\
    y_t^o &= \gamma + \delta P_t + v_t
  \end{cases}
\end{equation}
où $y_t^d$ et $y_t^o$ mesurent la demande et l'offre de bien, $P_t$ le
prix du bien, $u_t$ et $v_t$ sont des variables aléatoires iid,
centrées, de variances $\sigma_u^2$ et $\sigma_v^2$ et non corrélées, qui
s'interprètent comme des chocs de demande et d'offre. Le système
d'équation \ref{s1} est le modèle structurel. $\alpha$, $\beta$, $\gamma$ et $\delta$ les
paramètres structurels, $u_t$ et $v_t$ les chocs structurels. \newline

L'économètre observe les quantités échangées et les prix, on suppose
que le marché est équilibré à chaque date. \question Montrer qu'à
l'équilibre on doit avoir~:
\begin{equation}\label{s2}
  \begin{cases}
    y_t &= \mu_y + \varepsilon_{y,t}\\
    P_t &= \mu_P + \varepsilon_{P,t}\\
  \end{cases}
\end{equation}
en définissant explicitement $\mu_y$ et $\mu_P$ en fonction des
paramètres structurels, $\varepsilon_{y,t}$ et $\varepsilon_{P,t}$ en
fonction des chocs et paramètres structurels. Le système d'équation
\ref{s2} est le modèle réduit, $\mu_y$ et $\mu_P$ les paramètres
réduits, $\varepsilon_{y,t}$ et $\varepsilon_{P,t}$ les chocs
réduits. \question Est-il possible d'estimer sans biais les paramètres
réduits $\mu_y$ et $\mu_P$~? Donner les expressions des
estimateurs. \question Est-il possible de déduire les paramètres
structurels à partir des paramètres réduits~?\newline

Un économètre souhaite estimer la fonction de demande en régressant
les quantités échangées, $y_t$, sur les prix observés, $P_t$. Le
modèle empirique est~:
\[
y_t = a + b P_t + \epsilon_t, \quad t=1,\dots ,T
\]
\question Calculer la covariance entre $P_t$ et $u_t$. Que peut-on en
déduire sur les propriétés de l'estimateur des MCO~? \question
Calculer l'estimateur des MCO $\hat b$ et montrer que~:
\[
\text{plim}_{T\rightarrow\infty} \hat b = \frac{\sigma^2_v\beta+\sigma^2_u\delta}{\sigma^2_v+\sigma^2_u}
\]
Interpréter et commenter ce résultat.

\end{document}

%%% Local Variables:
%%% mode: latex
%%% TeX-master: t
%%% End:
